\documentclass[10pt]{article}
%{amsart}
\usepackage{graphicx,color}
%\usepackage{multibib}
\usepackage{epstopdf}
\usepackage{enumitem}
\usepackage{url}
%\usepackage{times}
%\usepackage{arial}
%\usepackage[margin=2cm]{geometry}
\usepackage{pgfgantt}
%\usepackage[default]{opensans}

\def\myidentsize {.5cm}
\usepackage{fullpage}

\usepackage{epstopdf}
   \DeclareGraphicsRule{.eps}{pdf}{.pdf}{`epstopdf #1}
   \pdfcompresslevel=9


\iffalse
\usepackage[tiny,bf]{titlesec}
\titleformat{\section}[hang]{\scshape\bfseries}{\thesection}{.5em}{}
\titlespacing{\section}{0pt}{0pt}{0pt}
\titleformat{\subsection}[hang]{\bfseries}{\thesubsection}{.2em}{}
\titleformat{\subsubsection}[runin]{\bfseries}{}{0pt}{}[.\quad]
\titleformat{\paragraph}[runin]{\itshape}{}{0pt}{}[.\quad]
\usepackage{fancyhdr,lastpage}
\lhead{}\chead{}\rhead{}
\lfoot{}\cfoot{Page \thepage\ of \pageref{LastPage}}\rfoot{}
\renewcommand{\headrule}{}
\pagestyle{fancy}
\fi

\newcommand{\B}{\color{blue}}
\newcommand{\D}{\color{purple}}
\newcommand{\heading}[1]{{\vspace{3pt}\noindent\sc{#1}}}
%\usepackage{fullpage}
%\topmargin -1cm
%\setlength\topmargin{.5cm}
%\setlength\oddsidemargin{2cm}
%\setlength\evensidemargin{2cm}
%\textheight 24cm
%\textwidth 15.5cm

%\usepackage{helvet}
\iffalse
\renewcommand{\familydefault}{\sfdefault}
  \usepackage[document]{ragged2e}
  \setlength\RaggedRightRightskip{0pt plus 1cm}
  \setlength\RaggedRightParindent{12pt}
\fi
\usepackage[a4paper,margin=20mm,footskip=6mm,headsep=3mm]{geometry}

%\usepackage[compact]{titlesec}
%\titleformat{\subsection}[runin]{\normalfont\bfseries}{\thesection}{1em}{}
%\titlespacing*{\subsection}{0pt}{2pt}{5pt}
%\titlespacing{\section}{2pt}{2pt}{2pt}

%\newcites{own}{Selected References for {B}ogdan {W}arinschi}

\newcommand{\bw}[1]{{\textcolor{blue} {Bogdan: #1}}}

\renewcommand{\thesection}{\Alph{section}}

\date{}

%\iffalse
\title{\bf Usable Security APIs\\
-- \ \ \ {\large Mental Models and Novel Abstractions \ \ \ } --}
\author{Awais Rashid and Bogdan Warinschi}

%\fi
%\title{}
%\author{}
\begin{document}
\maketitle
%\iffalse
\section{Thesis and supervisors}
%\subsection*{External Collaboration}


\paragraph*{Abstract}


\iffalse
\subsection*{Institute.}
The Computer Science Department at UoB is one of
the leading departments in the country, with $70$\% of research ranked as
$4\star$ or $3\star$ by RAE '08.
The Department enjoys close links with local and global industry, and maintains
a portfolio of ongoing research projects in collaboration with various
companies; it is also the base for the EPSRC funded LSCITS initiative.

The Cryptography and Information Security group at UoB is an acknowledged
world leader in the theory and implementation of cryptography, and is
one of the largest such groups in the world.
The group now consists of five permanent academic staff (in
addition to the PI): 
Prof.~Nigel Smart who works on theoretical and practical aspects of public key 
Cryptography especially elliptic curve based techniques, 
Dr.~Dan Page whose research is focused on tools and techniques to efficiently 
implement various cryptographic primitives in hardware and software, 
Dr.~Elisabeth Oswald who focuses on the physical security of computing devices, 
Dr.~Martijn Stam who works mainly on symmetric cryptography,  
and the related implementation issues, and 
Dr.~Theo Tryfonas who works in security engineering and forensics.
There are currently twelve PDRAs working on cryptography related projects, 
and sixteen PhD students.



The group was one of only two British academic representatives in ECRYPT, 
the European Framework 6 Network of Excellence in cryptography and is now a leading player in the Framework 7 successor  ECRYPT-II. 
Prof.~Smart sits on the Executive Management Committee of the network,
with Prof.~Smart and Dr.~Oswald also acting as working group co-leaders.
The network brings together the leading research groups from industry
and academia in the field of cryptography.
This solidifies the strong links the UoB group already has with our
leading peers in Europe, such as U.~Aarhus, RU.~Bochum, TU Darmstadt, TU Eindhoven, 
ENS Paris, HP Labs, IAIK, IBM Labs, KU Leuven, and Royal Holloway, University of London.


The University of Bristol has recently been acknowledged by 
EPSRC and GCHQ as an Academic Centre of Excellence in Cyber Security
Research. This status has been awarded to the University as a whole,
but the main constituent of the Centre is the Cryptography and Information
Security group.
\fi



\paragraph{Prof. Awais Rashid.}

\vspace{.5cm}


\paragraph{Prof. Bogdan Warinschi.}
Prof.~Warinschi, has joined the University of 
Bristol as a Lecturer in the Computer Science Department in February 2007. 
He has a Ph.D. in Computer Science from the University of California
at San Diego, and was previously affiliated as a postdoctoral researcher with
University of California at Santa Cruz, Stanford University, and the
French National Research Institute in Informatics (INRIA).
He conducts research into cryptography, with a particular emphasis on
proof methods for the security of protocols.


\iffalse
He was a co-investigator on the EP/H043454/1 EPSRC grant on "Privacy and Attestation Technologies" and on the ERC Advanced Grant ERC-2010-AdG-267188-CRIPTO on "Cryptography Research Involving Practical and Theoretical Outlooks". 
He is a co-PI on the FP 7 project ``PRACTICE" (grant agreement 609611).
%He currently acts as the university liaison with the Trusted Computing Group (TCG), a consortium of industrial stakeholders in charge with the development of the next generation of the TPM security co-processor.
\fi
Dr. Warinschi served on more than thirty program committees, most recently on Security and Privacy '11, CCS'11 and CCS'12. 
He was the programme chair of the Formal and Computational Cryptography (FCC'12) workshop. 
The workshop is dedicated to an area of cryptography called {\em computational   soundness} which Dr. Warinschi had helped establish~\cite{micciancio04completeness,micciancio04soundness,cortier05computationally,abadi05security,abadi05passwordbased,abadi06guessing,datta06computationally,cortier06computationally}.
The basic idea is to combine two fundamentally different paradigms for proving security, one based on symbolic (formal methods-like) techniques and one based on complexity and computation theory. The main benefit of the approach is that it enables simpler design of tools for computer-aided verification that yields computational security guarantees.
Part of the scientific underpining of this project relies on his expertise in this subject. 

In addition to his expertise in the area of computational soundness, this proposal relies on the expertise of Prof. Warinschi in the design of security models and his interest in analysing systems used in practise.
In particular, Dr. Warinschi has maintained a constant interest in designing cryptographic security models, the first step towards the rigorous analysis of any cryptographic system. 
His research addresses the security of group signatures \cite{BMW03}, proxy signatures \cite{proxies}, defence mechanisms for DOS resistance (cryptographic puzzles) \cite{puzzles},  key-exchange \cite{ke}, voting schemes \cite{helios}, and cryptographic APIs \cite{KSW11}.
This work is complemented by security analysis that led to a better understanding of protocols and primitives that are in current use.  These include Public Key Infrastructures (PKI) \cite{boldyreva07acloser}, the Trusted Platform Module (TPM) \cite{pcas}, the Helios Internet voting schemes \cite{helios}, and the ubiquitous Transport Layer Security protocol (TLS) \cite{tls}. 

%This proposal is perfectly aligned with Dr. Warinschi's expertise in the design of security models, his proficiency and interest in analysing systems used in practise, and his work on computational soundness for enabling tools for security analysis.


To date, seven students have completed their PhD under the (co-)supervision of Dr. Warinschi; he is currently (co-)supervising two 
students.


\paragraph{Collaboration.}
Prof. Rashid  ...


This set of skills makes him especially suited to advise on practical aspects of software development and human factors that impact developer.


Dr. Warinschi is a co-founder of the research area of computational soundness that links formal models and cryptographic security properties. Through his experience with cryptographic security definitions and proofs, he contributed to foundational work creating the security notions in use today for areas including group signatures and electronic voting. 
These factors make him an ideal supervisor for those aspects of this project which deal with abstractions and formal security models. 


\section{Context (2.5pages)}
\emph{There should be a clear statement that shows the proposed studentship will be addressing an important research challenge in cyber security. It should be clear that the supervisors have a good awareness of current technical approaches and that important parts of the cyber security research challenge are not being addressed by the current range of techniques available.}


\paragraph{Socio-economical context}
People and institutions surrender impressive amounts of sensitive data to a virtual space outside
their control.
%Society is highly dependent on virtual infrastructures that form what is generically known as the
%``cyberspace".
They purchase things on the web, communicate via cellular phones, store, share and utilize data kept on remote servers.
Technological advances (e.g. the proliferation of mobile devices) and the drive for increased
functionality (e.g. the advent of the Internet of Things, various health care initiatives) will
strengthen our dependency on ``cyberspace".
The number of connected devices reached 6 billion in 2008 and it is estimated to increase to 50
billion in the next five years.
This infrastructure is supported by software that is developed and deployed at amazing rates. 
Consider for example only the proliferation of mobile phone applications.
There are now 1.5 million applications in Google Play app store and about the same
number in Apple store.\footnote{\url{http://www.statista.com}}
Some of these mobile phones incorporate components that deal with sensitive user data, e.g. credit cards numbers.   

Given their increasingly important role and the opportunity for catastrophic failures that a
networked world entails\footnote{
Failures already lead to staggering amounts of trust, reputation, and monetary losses In 2011 EMC,
the parent company of RSA Inc., spent USD 66 Mil. to replace potentially compromised secure tokens.
} we need to hold applications against high security standards.
% In today's networked world, where attackers can remotely access devices and communication,
%Increased dependency and reliance on the cyberspace will greatly amplify the disruptive effects of
%failures.
%To prevent disaster, 
Regular reports of failures point to a perilous state of affairs. 
Between 2012 and 2014 the traffic of the ubiquitous messaging application \textsf{WhatsApp},
%currently by more than .7 billion users,
was encrypted with a symmetric key computable by any third
party.\footnote{\url{http://pastebin.com/g9UPuviz}}
The communication with \textsf{Amazon, PayPal, Rackspace, Chase} via some existing commercial
software can be easily intercepted by a malicious party~\cite{GIJABS12}.
Software written for major commercial players already fails more often than desired.  Unsurprisingly, the average
application fares much worse. 
A recent study found that 88\% of some eleven thousand Android applications misuse cryptography in
ways that violate the most basic best practice and standard
recommendations~\cite{egele13anempirical}.

\paragraph*{Problem overview.}

This should not be all that surprising.   Developers are rarely both skilled software engineers \emph{and} knowledgeable security experts.
Yet, they have to make decisions that require expertise at each of the different levels in the hierarchical structure of the cryptographic skeleton that sustains an application's security.
% offers numerous numerous opportunities for errors.
%The protocols employed, in turn build on the security of their underlying primitives. 
They need to select which cryptographic primitives to use, how to combine several cryptographic building blocks to obtain more advanced functionality and stronger guarantees, and ultimately have to integrate protocols themselves to fit together within the higher-level applications that employ them.


Consider for example a basic client-server application where the client delegates to the server some computation and where it is desired that the communication stays secret (from third parties).
An inexperienced designer would conclude that it suffices to simply encrypt the client-server communication. 
There are however several other crucial yet non-obvious questions that need to be answered before concluding that  encryption is sufficient. 
For example, the client needs to guarantee the identity of the server before it makes any query, encryptions should be authenticated~\cite{BN08}, replay of messages that were sent encrypted should not be possible, etc. 

Employing protocols (rather than vanilla encryption) goes a long way towards avoiding many subtle pitfalls.  
For example, when used with care, a secure channel protocol like the Transport Layer Security (TLS) takes care of entity and message authentication, integrity of the communication and so on.  Furthermore, good, widely vetted protocol implementations are readily available through libraries.  However, using these libraries faces a designer/implementer with further challenges.   The APIs of these libraries are complex and unintuitive, require users to set often obscure parameters, and select between many possible configurations -- some secure, some insecure!  Many attacks are in fact due to incompetent uses of existing libraries  \cite{GIJABS12}. 

\paragraph{Mismatch between developers understanding of APIs}
The discussion highlights various mismatches between how developers perceive the security delivered by cryptographic constructs (as available, e.g. through the APIs of standard cryptographic libraries) and the guarantees that these actually offer. 

We note that while there is a lot of work on mental models of \emph{end-users} there is  none that we are aware of on mental models of \emph{developers}. 





\paragraph{Existent work.} We briefly survey existent work in this area, and highlight some of their drawbacks on which we plan to improve upon.

%\clearpage

\section{Proposed research and technical approach (2.5 pages)}

\emph{The technical approach proposed should be novel with the potential for a doctoral student to make a scholarly contribution to the technical area. The research area must be included in the list of research areas within the scope of the Call listed in Appendix A. The proposed milestones should be ambitious but realistic.}




%

\subsection{Research hypothesis and objectives.}


There is by now sufficient evidence that, many of common errors that developers make are due to a mismatch between 






\subsection{Project structure and methodology}


We structure the research in the project along two distinct direction: a study of mental models of developers with respect to security APIs and the development of novel abstractions for security functionality.  These two research directions naturaly translate to a corresponding workpackage. 
A third workpackage 




\begin{description}\itemsep-3pt
\item[WP 1] Mental models for developers with respect to security APIs and composition. 
We will start with a study of mental models for software developers which will explain how they perceive and utilise the interface of typical security APIs (e.g. OAuth and TLS). 

\bw{from email exchange:}
We can study how well the mental models *align* with what is a *correct* model.   We can analyse how far incorrect mental models are from what would be a correct one. 

\subparagraph{Milestones.} 

\item[WP 2] New abstractions. 

In this work package we will use the results of WP1 to develop improved interfaces to the libraries that implement security features. 
Specifically, we will use the mental models developed in WP1 to identify gaps between how developers perceive the security guarantees offered by library components and the actualy guarantees that are offered. 
These gaps will also suggest novel abstractions which expose to developers functionality which is better aligned with their peception of security functionality, as captured by the models developed in the previous workpackage. 

We will provide both theoretical grounding (in the form of variations of existing security constructs or entirely new ones) and practical realizations in the form of improved APIs which reflect the new abstractions which we develop.  For instance, we envision that an implementer will be able to establish a secure channel between two parties by invoking a high-level command that hides almost entirely the underlying cryptographic mechanisms.  
In particular, employing TLS will not require the implementer to set up parameters or ensure certificate validation.  
Technically, we will provide wrappers to some of the most commonly used libraries.  
We remark that improved APIs is one of the feature of the NaCl cryptographic library~\cite{bernstein}; currently that library only considers lower level primitives.  
\item[WP 3] Evaluation

The evaluation workpackage will run alongside and act as liason between workpackages WP1 and WP2.  
We envision an iterative process where the mental models developed in WP1 inspire the abstractions developed in WP2, which on the one hand can be benchmarked against the models and on the other lead to more refined models. 

\vspace{-\baselineskip}
\subparagraph{Milestones.}


\end{description}



\subparagraph{Milestones.}



\paragraph{WP3 -- Evaluation.}  

%We summarise the workplan in the following chart:

%\begin{center}
%\begin{ganttchart}[hgrid,vgrid]{1}{21}
%\gantttitle{year 1}{6}
%\gantttitle{year 2}{6}
%\gantttitle{year 3}{6}
%\gantttitle{year 4}{3}
%\ganttnewline
%\gantttitle{6 mo}{3}
%\gantttitle{12 mo}{3}
%\gantttitle{18 mo}{3}
%\gantttitle{24 mo}{3}
%\gantttitle{30 mo}{3}
%\gantttitle{36 mo}{3}
%\gantttitle{42 mo}{3}

%\ganttnewline

%\ganttbar{preparation}{1}{3}

%\ganttnewline

%\ganttbar{WP 1}{4}{9}
%\ganttmilestone{}{6}

%\ganttnewline

%\ganttbar{WP 2}{10}{15}
%\ganttmilestone{}{12}

%\ganttnewline

%\ganttbar{WP 3}{16}{18}
%\ganttmilestone{}{18}

%\ganttnewline

%\ganttbar{thesis}{19}{21}

%\end{ganttchart}
%\end{center}


\subsection*{Alignment with Research Areas in the Call}

The proposed studentship is aligned with several of the research directions outlined in the call for proposals. The more pertinent ones are the following. 

\begin{itemize}
\item ``8. Building trusted and trustworthy systems'' through ``improved methods for the development or assessment of secure system' and also 
``6. Strategic Technologies and Products" specifically via ''Research into the quality, prevalence, variation and utilisation of mitigation/safety features in coding platforms, especially given the uplift in new languages.'' 

Our proposal addresses these issue head on. We seek to develop inproved APIs to security libraries which expose to developers abstractions that are better allighned with their intuition of the underlying security mechanism.  Such interfaces should facilitate software development which employs security features in ways that are less error-prone. 
\item \bw{Perhaps a bit of a stretch} The project also intersects with  "2. Human-Centered Security and Security Analytics": Developer mental models can be seen as a way of capturing (potentially in a formal way) the culture around usage of security interfaces albeit by seeing the developer as a user of these interfaces. 
\item 
\end{itemize} 








 


\section{Costs}
The costs associated to this project are summarized in Figure~\ref{fig:costs}.  
The costs associated to travel and subsistence include participation at conferences and summer schools for project members.
\bw{I've added some completely made up costs for User study costs. We need to converge soon on these so that I can add them to the costing tool}
\begin{figure}[h!t]
\begin{center}
\begin{tabular}{|c|c|c|c|c|}
\hline 
& {\bf Year 1} &{\bf Year 2}&{\bf Year 3}&{\bf Year 4}\\
\hline
Fees & £4195 & £4290 & £4387 & 2243 \\
\hline
GCHQ stipend & £22000 & £22500 & £23000 & £11750 \\
\hline
Travel and subsistence & 5000 & 5000 & 5000 & 2500 \\
\hline
User study costs & 2500 & 2500 & 2500 & 1000 \\
%\hline
%Other costs & 2500 & - & - & - \\
\hline
\end{tabular}
\end{center}
\caption{Summary of the project costs}
\label{fig:costs}
\end{figure}

%\bibliography{biblio}
%\bibliographystyle{plain}
%\renewcommand{\refname}{References}
%\input{apis-bbl}
%\bibliographystyle{plain}
\small
\bibliographystyle{abbrv}
\bibliography{refs}

\end{document}
